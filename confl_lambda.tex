\coqlibrary{confl lambda}{Library }{confl\_lambda}

\begin{coqdoccode}
\coqdocemptyline
\end{coqdoccode}
\section{Locally Nameless Notation}

\begin{coqdoccode}
\coqdocemptyline
\coqdocemptyline
\end{coqdoccode}
A notação de nomes locais é definida de forma a trabalhar com index's ligados à abstrações
para representar variáveis ligadas, e variáveis nomeadas para representar variáveis livres. Dessa
forma podemos ter um conjunto de símbolos nesta notação que não sejam válidos semanticamente. 
Vamos exemplificar: 

 Dada a abstração $\lambda.0$ sabemos que o index 0 representa a variável ligada à abstração
que apresentamos. O index 0 neste caso indica que existem 0 passos para em uma aplicação
substituir a variável ligada que o mesmo representa, estando esse index portanto ligado
diretamente à abstração apresentada, e sendo dessa maneira válido semanticamente. Porém,
sintaticamente o index pode ser um valor iteiro k qualquer. Assim, por exemplo, ao 
fazermos $\lambda.1$, estamos construindo semanticamente uma relação que não possui 
significado válido. Queremos representar variáveis ligadas por meio de index's e neste caso o 
index 1 não está ligado à nenhuma abstração, portanto não representa uma variável ligada
e não é válido para nossa representação.

 Assim podemos definir o conceito de pré-termo como sendo: Todo conjunto de símbolos sintaticamente
válidos, que ainda não temos certeza acerca da validade semântica. Matematicamente os pré-termos 
são definidos de acordo com a seguinte gramática: \begin{coqdoccode}
\coqdocemptyline
\coqdocnoindent
\coqdockw{Inductive} \coqdocvar{pterm} : \coqdockw{Set} :=\coqdoceol
\coqdocindent{1.00em}
\ensuremath{|} \coqdocvar{pterm\_bvar} : \coqdocvar{nat} \ensuremath{\rightarrow} \coqdocvar{pterm}\coqdoceol
\coqdocindent{1.00em}
\ensuremath{|} \coqdocvar{pterm\_fvar} : \coqdocvar{var} \ensuremath{\rightarrow} \coqdocvar{pterm}\coqdoceol
\coqdocindent{1.00em}
\ensuremath{|} \coqdocvar{pterm\_app}  : \coqdocvar{pterm} \ensuremath{\rightarrow} \coqdocvar{pterm} \ensuremath{\rightarrow} \coqdocvar{pterm}\coqdoceol
\coqdocindent{1.00em}
\ensuremath{|} \coqdocvar{pterm\_abs}  : \coqdocvar{pterm} \ensuremath{\rightarrow} \coqdocvar{pterm}\coqdoceol
\coqdocindent{1.00em}
\ensuremath{|} \coqdocvar{pterm\_labs}  : \coqdocvar{pterm} \ensuremath{\rightarrow} \coqdocvar{pterm}.\coqdoceol
\coqdocemptyline
\end{coqdoccode}
Um pré-termo é um termo do cálculo lambda que possui em si indexs de de Bruijin que não foram
substituídos. (esta definição está correta?)

Definição alternativa:

 Um pré-termo é um conjunto de símbolos do cálculo lambda que operam conjuntamente e ainda
não foram verificados como totalmente fechados.\begin{coqdoccode}
\coqdocemptyline
\coqdocemptyline
\end{coqdoccode}
A definição da operação ``variable opening'', ou seja, abertura de variáveis é dada abaixo:\begin{coqdoccode}
\coqdocemptyline
\coqdocnoindent
\coqdockw{Fixpoint} \coqdocvar{open\_rec} (\coqdocvar{k} : \coqdocvar{nat}) (\coqdocvar{u} : \coqdocvar{pterm}) (\coqdocvar{t} : \coqdocvar{pterm}) : \coqdocvar{pterm} :=\coqdoceol
\coqdocindent{1.00em}
\coqdockw{match} \coqdocvar{t} \coqdockw{with}\coqdoceol
\coqdocindent{1.00em}
\ensuremath{|} \coqdocvar{pterm\_bvar} \coqdocvar{i}    \ensuremath{\Rightarrow} \coqdockw{if} \coqdocvar{k} === \coqdocvar{i} \coqdockw{then} \coqdocvar{u} \coqdockw{else} (\coqdocvar{pterm\_bvar} \coqdocvar{i})\coqdoceol
\coqdocindent{1.00em}
\ensuremath{|} \coqdocvar{pterm\_fvar} \coqdocvar{x}    \ensuremath{\Rightarrow} \coqdocvar{pterm\_fvar} \coqdocvar{x}\coqdoceol
\coqdocindent{1.00em}
\ensuremath{|} \coqdocvar{pterm\_app} \coqdocvar{t1} \coqdocvar{t2} \ensuremath{\Rightarrow} \coqdocvar{pterm\_app} (\coqdocvar{open\_rec} \coqdocvar{k} \coqdocvar{u} \coqdocvar{t1}) (\coqdocvar{open\_rec} \coqdocvar{k} \coqdocvar{u} \coqdocvar{t2})\coqdoceol
\coqdocindent{1.00em}
\ensuremath{|} \coqdocvar{pterm\_abs} \coqdocvar{t1}    \ensuremath{\Rightarrow} \coqdocvar{pterm\_abs} (\coqdocvar{open\_rec} (\coqdocvar{S} \coqdocvar{k}) \coqdocvar{u} \coqdocvar{t1})\coqdoceol
\coqdocindent{1.00em}
\ensuremath{|} \coqdocvar{pterm\_labs} \coqdocvar{t1}    \ensuremath{\Rightarrow} \coqdocvar{pterm\_labs} (\coqdocvar{open\_rec} (\coqdocvar{S} \coqdocvar{k}) \coqdocvar{u} \coqdocvar{t1})\coqdoceol
\coqdocindent{1.00em}
\coqdockw{end}.\coqdoceol
\coqdocemptyline
\end{coqdoccode}
Esta operação é responsável por substituis todos os índices ``k'', por uma variável com
nome qualquer. Por exemplo, digamos que tenhamos o pré-termo $\lambda.0 y$, assim, ao aplicar
a operação $ {0 ~> x} \lambda.0 y$ teremos o seguinte resultado: $\lambda.x y$.

 Com isso, a operação de abertura recursiva apenas para index's 0 é definida especialmente 
como ``open'', onde u é o nome de uma variável qualquer e t é um pré-termo ,logo abaixo:\begin{coqdoccode}
\coqdocemptyline
\coqdocnoindent
\coqdockw{Definition} \coqdocvar{open} \coqdocvar{t} \coqdocvar{u} := \coqdocvar{open\_rec} 0 \coqdocvar{u} \coqdocvar{t}.\coqdoceol
\coqdocemptyline
\end{coqdoccode}
Esta definição será extremamente útil nas provas mais abaixo, devido à maior facilidade com relação
à trabalhar com qualquer index k.

 De qualquer forma, a operação como explicada mais acima, onde k é o index de De Bruijin,
u o nome de uma variável qualquer e t o pré-termo que será aberto recursivamente no index k é 
definida como se segue:\begin{coqdoccode}
\coqdocemptyline
\coqdocnoindent
\coqdockw{Notation} "\{ k \~{}> u \} t" := (\coqdocvar{open\_rec} \coqdocvar{k} \coqdocvar{u} \coqdocvar{t}) (\coqdoctac{at} \coqdockw{level} 67).\coqdoceol
\coqdocemptyline
\coqdocemptyline
\end{coqdoccode}
Notação para representar a abertura de um termo, substituindo as variáveis ligadas
por qualquer tipo de pré-termo.\begin{coqdoccode}
\coqdocemptyline
\coqdocnoindent
\coqdockw{Notation} "t \^{}\^{} u" := (\coqdocvar{open} \coqdocvar{t} \coqdocvar{u}) (\coqdoctac{at} \coqdockw{level} 67).\coqdoceol
\coqdocemptyline
\end{coqdoccode}
Notação para representar abertura de um pré-termo utilizando uma variável livre
x:\begin{coqdoccode}
\coqdocemptyline
\coqdocnoindent
\coqdockw{Notation} "t \^{} x" := (\coqdocvar{open} \coqdocvar{t} (\coqdocvar{pterm\_fvar} \coqdocvar{x})).\coqdoceol
\coqdocemptyline
\coqdocemptyline
\end{coqdoccode}
A definição de termo é dada logo abaixo:\begin{coqdoccode}
\coqdocemptyline
\coqdocnoindent
\coqdockw{Inductive} \coqdocvar{term} : \coqdocvar{pterm} \ensuremath{\rightarrow} \coqdockw{Prop} :=\coqdoceol
\coqdocindent{1.00em}
\ensuremath{|} \coqdocvar{term\_var} : \coqdockw{\ensuremath{\forall}} \coqdocvar{x},\coqdoceol
\coqdocindent{3.00em}
\coqdocvar{term} (\coqdocvar{pterm\_fvar} \coqdocvar{x})\coqdoceol
\coqdocindent{1.00em}
\ensuremath{|} \coqdocvar{term\_app} : \coqdockw{\ensuremath{\forall}} \coqdocvar{t1} \coqdocvar{t2},\coqdoceol
\coqdocindent{3.00em}
\coqdocvar{term} \coqdocvar{t1} \ensuremath{\rightarrow} \coqdoceol
\coqdocindent{3.00em}
\coqdocvar{term} \coqdocvar{t2} \ensuremath{\rightarrow} \coqdoceol
\coqdocindent{3.00em}
\coqdocvar{term} (\coqdocvar{pterm\_app} \coqdocvar{t1} \coqdocvar{t2})\coqdoceol
\coqdocindent{1.00em}
\ensuremath{|} \coqdocvar{term\_abs} : \coqdockw{\ensuremath{\forall}} \coqdocvar{L} \coqdocvar{t1},\coqdoceol
\coqdocindent{3.00em}
(\coqdockw{\ensuremath{\forall}} \coqdocvar{x}, \coqdocvar{x} \symbol{92}\coqdocvar{notin} \coqdocvar{L} \ensuremath{\rightarrow} \coqdocvar{term} (\coqdocvar{t1} \^{} \coqdocvar{x})) \ensuremath{\rightarrow}\coqdoceol
\coqdocindent{3.00em}
\coqdocvar{term} (\coqdocvar{pterm\_abs} \coqdocvar{t1}).\coqdoceol
\coqdocemptyline
\end{coqdoccode}
Um termo é válido na notação de nomes locais, quando este termo é um termo fechado. 
Para ser um termo fechado, seguimos uma definição recursiva, onde toda variável livre, e portanto
nomeada é fechada, toda aplicação é fechada, se seus dois termos internos são fechados, e toda
abstração é fechada, se todos os termos que fazem parte da mesma também são fechados. Veja que esta é
exatamente a definição que temos logo acima. Podemos deixar o entendimento mais fácil com a seguinte
definição: Um termo é um pré-termo sem nenhuma variável ligada inválida, ou seja, nenhum index não ligado
à uma abstração de alguma maneira.

 Dessa forma também é interessante definir o conceito de corpo, como sendo todo pré-termo
que após uma abertura no index 0, por uma variável nomeada livre x, torna-se um termo fechado.
A definição está logo abaixo:\begin{coqdoccode}
\coqdocemptyline
\coqdocnoindent
\coqdockw{Definition} \coqdocvar{body} \coqdocvar{t} :=\coqdoceol
\coqdocindent{1.00em}
\coqdoctac{\ensuremath{\exists}} \coqdocvar{L}, \coqdockw{\ensuremath{\forall}} \coqdocvar{x}, \coqdocvar{x} \symbol{92}\coqdocvar{notin} \coqdocvar{L} \ensuremath{\rightarrow} \coqdocvar{term} (\coqdocvar{t} \^{} \coqdocvar{x}).\coqdoceol
\coqdocemptyline
\end{coqdoccode}
Perceba que a definição de body foi utilizada na definição recursiva de termo, para
representar as abtrações válidas semanticamente, segundo os conceitos aqui já apresentados.\begin{coqdoccode}
\coqdocemptyline
\coqdocemptyline
\end{coqdoccode}
Para realizar a prova da confluência é necessária a definição de um termo marcado. Um
termo marcado contém exatamente as mesmas propriedades de um termo, exceto que, pode possuir
abstrações marcadas. Abstrações marcadas são aquelas que estão postas em uma aplicação válida,
aonde pode ser aplicada uma B-redução. Abstrações que não fazem parte de uma aplicação, não podem ser 
abstrações marcadas. Podemos ver essa definição em termos recursivos logo abaixo:\begin{coqdoccode}
\coqdocemptyline
\coqdocnoindent
\coqdockw{Inductive} \coqdocvar{lterm} : \coqdocvar{pterm} \ensuremath{\rightarrow} \coqdockw{Prop} :=\coqdoceol
\coqdocindent{1.00em}
\ensuremath{|} \coqdocvar{lterm\_var} : \coqdockw{\ensuremath{\forall}} \coqdocvar{x},\coqdoceol
\coqdocindent{3.00em}
\coqdocvar{lterm} (\coqdocvar{pterm\_fvar} \coqdocvar{x})\coqdoceol
\coqdocindent{1.00em}
\ensuremath{|} \coqdocvar{lterm\_app} : \coqdockw{\ensuremath{\forall}} \coqdocvar{t1} \coqdocvar{t2},\coqdoceol
\coqdocindent{3.00em}
\coqdocvar{lterm} \coqdocvar{t1} \ensuremath{\rightarrow} \coqdoceol
\coqdocindent{3.00em}
\coqdocvar{lterm} \coqdocvar{t2} \ensuremath{\rightarrow} \coqdoceol
\coqdocindent{3.00em}
\coqdocvar{lterm} (\coqdocvar{pterm\_app} \coqdocvar{t1} \coqdocvar{t2})\coqdoceol
\coqdocindent{1.00em}
\ensuremath{|} \coqdocvar{lterm\_abs} : \coqdockw{\ensuremath{\forall}} \coqdocvar{L} \coqdocvar{t1},\coqdoceol
\coqdocindent{3.00em}
(\coqdockw{\ensuremath{\forall}} \coqdocvar{x}, \coqdocvar{x} \symbol{92}\coqdocvar{notin} \coqdocvar{L} \ensuremath{\rightarrow} \coqdocvar{lterm} (\coqdocvar{t1} \^{} \coqdocvar{x})) \ensuremath{\rightarrow}\coqdoceol
\coqdocindent{3.00em}
\coqdocvar{lterm} (\coqdocvar{pterm\_abs} \coqdocvar{t1})\coqdoceol
\coqdocindent{1.00em}
\ensuremath{|} \coqdocvar{lterm\_labs} : \coqdockw{\ensuremath{\forall}} \coqdocvar{L} \coqdocvar{t1} \coqdocvar{t2},\coqdoceol
\coqdocindent{3.00em}
(\coqdockw{\ensuremath{\forall}} \coqdocvar{x}, \coqdocvar{x} \symbol{92}\coqdocvar{notin} \coqdocvar{L} \ensuremath{\rightarrow} \coqdocvar{lterm} (\coqdocvar{t1} \^{} \coqdocvar{x})) \ensuremath{\rightarrow}\coqdoceol
\coqdocindent{3.00em}
\coqdocvar{lterm} \coqdocvar{t2} \ensuremath{\rightarrow}\coqdoceol
\coqdocindent{3.00em}
\coqdocvar{lterm} (\coqdocvar{pterm\_app} (\coqdocvar{pterm\_labs} \coqdocvar{t1}) \coqdocvar{t2}).\coqdoceol
\coqdocemptyline
\end{coqdoccode}
Assim, também é possível definir o corpo marcado, como sendo o pré-termo que possui dentro de 
si uma abstração marcada, aonde após uma abertura recursiva do index 0 com uma variável nomeada 
livre qualquer x, torna-se um termo marcado:\begin{coqdoccode}
\coqdocemptyline
\coqdocnoindent
\coqdockw{Definition} \coqdocvar{lbody} \coqdocvar{t} :=\coqdoceol
\coqdocindent{1.00em}
\coqdoctac{\ensuremath{\exists}} \coqdocvar{L}, \coqdockw{\ensuremath{\forall}} \coqdocvar{x}, \coqdocvar{x} \symbol{92}\coqdocvar{notin} \coqdocvar{L} \ensuremath{\rightarrow} \coqdocvar{lterm} (\coqdocvar{t} \^{} \coqdocvar{x}).\coqdoceol
\coqdocemptyline
\coqdocnoindent
\coqdockw{Hint Constructors} \coqdocvar{lterm} \coqdocvar{term}.\coqdoceol
\coqdocemptyline
\coqdocemptyline
\coqdocemptyline
\end{coqdoccode}
*Definições características 

 Para ajudar nas diversas abordagens de prova que serão utilizadas, podemos definir
algumas propriedades. Em provas indutivas no tamanho da estrutura sintática, precisamos 
definir o conceito de tamanho do termo. Esta definição é dada abaixo, dando o valor de 1
para variáveis livres (pterm\_fvar x) e variáveis ligadas (pterm\_bvar i) e contando recursivamente
a partir das estruturas mais complexas do termo, tal como a aplicação, a abstração e a abstração
marcada: \begin{coqdoccode}
\coqdocemptyline
\coqdocnoindent
\coqdockw{Fixpoint} \coqdocvar{pterm\_size} (\coqdocvar{t} : \coqdocvar{pterm}) : \coqdocvar{nat} :=\coqdoceol
\coqdocindent{0.50em}
\coqdockw{match} \coqdocvar{t} \coqdockw{with}\coqdoceol
\coqdocindent{0.50em}
\ensuremath{|} \coqdocvar{pterm\_bvar} \coqdocvar{i}    \ensuremath{\Rightarrow} 1\coqdoceol
\coqdocindent{0.50em}
\ensuremath{|} \coqdocvar{pterm\_fvar} \coqdocvar{x}    \ensuremath{\Rightarrow} 1\coqdoceol
\coqdocindent{0.50em}
\ensuremath{|} \coqdocvar{pterm\_app} \coqdocvar{t1} \coqdocvar{t2} \ensuremath{\Rightarrow} (\coqdocvar{pterm\_size} \coqdocvar{t1}) + (\coqdocvar{pterm\_size} \coqdocvar{t2})\coqdoceol
\coqdocindent{0.50em}
\ensuremath{|} \coqdocvar{pterm\_abs} \coqdocvar{t1}    \ensuremath{\Rightarrow} 1 + (\coqdocvar{pterm\_size} \coqdocvar{t1})\coqdoceol
\coqdocindent{0.50em}
\ensuremath{|} \coqdocvar{pterm\_labs} \coqdocvar{t1}   \ensuremath{\Rightarrow} 1 + (\coqdocvar{pterm\_size} \coqdocvar{t1})\coqdoceol
\coqdocindent{0.50em}
\coqdockw{end}.\coqdoceol
\coqdocemptyline
\coqdocnoindent
\coqdockw{Lemma} \coqdocvar{pterm\_size\_gt\_0}: \coqdockw{\ensuremath{\forall}} \coqdocvar{t}, \coqdocvar{pterm\_size} \coqdocvar{t} > 0.\coqdoceol
\coqdocnoindent
\coqdockw{Proof}.\coqdoceol
\coqdocindent{1.00em}
\coqdoctac{intro} \coqdocvar{t}; \coqdoctac{induction} \coqdocvar{t}.\coqdoceol
\coqdocindent{1.00em}
- \coqdoctac{simpl}.\coqdoceol
\coqdocindent{2.00em}
\coqdoctac{auto}.\coqdoceol
\coqdocindent{1.00em}
- \coqdoctac{simpl}.\coqdoceol
\coqdocindent{2.00em}
\coqdoctac{auto}.\coqdoceol
\coqdocindent{1.00em}
- \coqdoctac{simpl}.\coqdoceol
\coqdocindent{2.00em}
\coqdoctac{apply} \coqdocvar{Nat.add\_pos\_r}; \coqdoctac{assumption}.\coqdoceol
\coqdocindent{1.00em}
- \coqdoctac{simpl}.\coqdoceol
\coqdocindent{2.00em}
\coqdoctac{auto}.\coqdoceol
\coqdocindent{1.00em}
- \coqdoctac{simpl}.\coqdoceol
\coqdocindent{2.00em}
\coqdoctac{auto}.\coqdoceol
\coqdocnoindent
\coqdockw{Qed}.\coqdoceol
\coqdocemptyline
\coqdocnoindent
\coqdockw{Lemma} \coqdocvar{pterm\_size\_open\_rec}: \coqdockw{\ensuremath{\forall}} \coqdocvar{t} \coqdocvar{n} \coqdocvar{x}, \coqdocvar{pterm\_size} \coqdocvar{t} = \coqdocvar{pterm\_size} (\coqdocvar{open\_rec} \coqdocvar{n} (\coqdocvar{pterm\_fvar} \coqdocvar{x}) \coqdocvar{t}).\coqdoceol
\coqdocnoindent
\coqdockw{Proof}.\coqdoceol
\coqdocindent{1.00em}
\coqdoctac{intro} \coqdocvar{t}; \coqdoctac{induction} \coqdocvar{t}.\coqdoceol
\coqdocindent{1.00em}
- \coqdoctac{intros} \coqdocvar{n0} \coqdocvar{x}.\coqdoceol
\coqdocindent{2.00em}
\coqdoctac{simpl}.\coqdoceol
\coqdocindent{2.00em}
\coqdoctac{destruct}(\coqdocvar{n0} === \coqdocvar{n}); \coqdoctac{reflexivity}.\coqdoceol
\coqdocindent{1.00em}
- \coqdoctac{intros} \coqdocvar{n} \coqdocvar{x}.\coqdoceol
\coqdocindent{2.00em}
\coqdoctac{reflexivity}.\coqdoceol
\coqdocindent{1.00em}
- \coqdoctac{intros} \coqdocvar{n} \coqdocvar{x}.\coqdoceol
\coqdocindent{2.00em}
\coqdoctac{simpl}.\coqdoceol
\coqdocindent{2.00em}
\coqdoctac{rewrite} \ensuremath{\leftarrow} \coqdocvar{IHt1}.\coqdoceol
\coqdocindent{2.00em}
\coqdoctac{rewrite} \ensuremath{\leftarrow} \coqdocvar{IHt2}.\coqdoceol
\coqdocindent{2.00em}
\coqdoctac{reflexivity}.\coqdoceol
\coqdocindent{1.00em}
- \coqdoctac{intros} \coqdocvar{n} \coqdocvar{x}.\coqdoceol
\coqdocindent{2.00em}
\coqdoctac{simpl}.\coqdoceol
\coqdocindent{2.00em}
\coqdoctac{rewrite} \ensuremath{\leftarrow} \coqdocvar{IHt}.\coqdoceol
\coqdocindent{2.00em}
\coqdoctac{reflexivity}.\coqdoceol
\coqdocindent{1.00em}
- \coqdoctac{intros} \coqdocvar{n} \coqdocvar{x}.\coqdoceol
\coqdocindent{2.00em}
\coqdoctac{simpl}.\coqdoceol
\coqdocindent{2.00em}
\coqdoctac{rewrite} \ensuremath{\leftarrow} \coqdocvar{IHt}.\coqdoceol
\coqdocindent{2.00em}
\coqdoctac{reflexivity}.\coqdoceol
\coqdocnoindent
\coqdockw{Qed}.\coqdoceol
\coqdocemptyline
\coqdocemptyline
\end{coqdoccode}
\section{Seção}



\subsection{Subseção}



 texto do relatório \begin{coqdoccode}
\coqdocemptyline
\coqdocnoindent
\coqdockw{Theorem} \coqdocvar{strip\_lemma}: \coqdockw{\ensuremath{\forall}}  \coqdocvar{t} \coqdocvar{t1} \coqdocvar{t2}, \coqdocvar{t} -->\coqdocvar{B} \coqdocvar{t1} \ensuremath{\rightarrow} \coqdocvar{t} -->>\coqdocvar{B} \coqdocvar{t2} \ensuremath{\rightarrow} \coqdoctac{\ensuremath{\exists}} \coqdocvar{t3}, \coqdocvar{t1} -->>\coqdocvar{B} \coqdocvar{t3} \ensuremath{\land} \coqdocvar{t2} -->>\coqdocvar{B} \coqdocvar{t3}.\coqdoceol
\coqdocindent{1.00em}
\coqdockw{Proof}.\coqdoceol
\coqdocindent{1.00em}
\coqdoctac{intros} \coqdocvar{t} \coqdocvar{t1} \coqdocvar{t2} \coqdocvar{H1} \coqdocvar{H2}.\coqdoceol
\coqdocindent{1.00em}
\coqdoctac{induction} \coqdocvar{H1}.\coqdoceol
\coqdocindent{1.00em}
- \coqdoctac{inversion} \coqdocvar{H}; \coqdoctac{subst}.\coqdoceol
\coqdocindent{2.00em}
\coqdoctac{apply} \coqdocvar{refltrans\_equiv} \coqdoctac{in} \coqdocvar{H2}.\coqdoceol
\coqdocindent{2.00em}
\coqdocvar{remember} (\coqdocvar{t1} \^{}\^{} \coqdocvar{u}) \coqdockw{as} \coqdocvar{b}.\coqdoceol
\coqdocindent{2.00em}
\coqdoctac{generalize}  \coqdoctac{dependent} \coqdocvar{b}.\coqdoceol
\coqdocindent{2.00em}
\coqdoctac{induction} \coqdocvar{H2}.\coqdoceol
\coqdocindent{2.00em}
+ \coqdoctac{\ensuremath{\exists}} (\coqdocvar{t1} \^{}\^{} \coqdocvar{u}); \coqdoctac{split}.\coqdoceol
\coqdocindent{3.00em}
\ensuremath{\times} \coqdoctac{apply} \coqdocvar{reflex}.\coqdoceol
\coqdocindent{3.00em}
\ensuremath{\times} \coqdoctac{apply} \coqdocvar{atleast1}.\coqdoceol
\coqdocindent{4.00em}
\coqdoctac{apply} \coqdocvar{redex}.\coqdoceol
\coqdocindent{4.00em}
\coqdoctac{assumption}.\coqdoceol
\coqdocindent{2.00em}
+ \coqdoctac{inversion} \coqdocvar{H}; \coqdoctac{subst}.\coqdoceol
\coqdocindent{3.00em}
\coqdoctac{rewrite} \ensuremath{\leftarrow} \coqdocvar{H3} \coqdoctac{in} \coqdocvar{H}.\coqdoceol
\coqdocindent{3.00em}
\coqdoctac{clear} \coqdocvar{H0} \coqdocvar{H1} \coqdocvar{H3}.\coqdoceol
\coqdocemptyline
\coqdocindent{3.00em}
\coqdoctac{inversion} \coqdocvar{H2}; \coqdoctac{subst}.\coqdoceol
\coqdocindent{3.00em}
\ensuremath{\times} \coqdocvar{admit}.\coqdoceol
\coqdocindent{3.00em}
\ensuremath{\times} \coqdocvar{admit}.\coqdoceol
\coqdocindent{3.00em}
\ensuremath{\times} \coqdocvar{admit}.\coqdoceol
\coqdocindent{2.00em}
+ \coqdoceol
\coqdocindent{3.00em}
\coqdoctac{apply} \coqdocvar{open\_rec\_inj} \coqdoctac{in} \coqdocvar{H3}.\coqdoceol
\coqdocindent{3.00em}
\coqdoctac{destruct} \coqdocvar{H3} \coqdockw{as} [\coqdocvar{Heq1} \coqdocvar{Heq2}]; \coqdoctac{subst}.\coqdoceol
\coqdocindent{3.00em}
\coqdoctac{clear} \coqdocvar{H4} \coqdocvar{H6}.\coqdoceol
\coqdocindent{3.00em}
\coqdoctac{induction} \coqdocvar{H2}.\coqdoceol
\coqdocindent{3.00em}
\ensuremath{\times} \coqdocvar{Admitted}.\coqdoceol
\coqdocemptyline
\coqdocemptyline
\coqdocindent{2.00em}
\coqdockw{Lemma} \coqdocvar{erase\_lbeta\_2313}: \coqdockw{\ensuremath{\forall}} \coqdocvar{t1} \coqdocvar{t2}, \coqdocvar{t1} -->>\coqdocvar{B} \coqdocvar{t2} \ensuremath{\rightarrow} \coqdockw{\ensuremath{\forall}} \coqdocvar{t1'} \coqdocvar{t2'}, \coqdocvar{erase}(\coqdocvar{t1'}) = \coqdocvar{t1} \ensuremath{\land} \coqdocvar{erase}(\coqdocvar{t2'}) = \coqdocvar{t2} \ensuremath{\rightarrow} \coqdocvar{t1'} -->>\coqdocvar{lB} \coqdocvar{t2'}.\coqdoceol
\coqdocindent{2.00em}
\coqdockw{Proof}.\coqdoceol
\coqdocindent{3.00em}
\coqdocvar{Admitted}.\coqdoceol
\coqdocemptyline
\coqdocindent{2.00em}
\coqdoctac{assert} (\coqdocvar{H'}: \coqdocvar{pterm\_app} (\coqdocvar{pterm\_labs} \coqdocvar{t1}) \coqdocvar{u} -->>\coqdocvar{B} \coqdocvar{t2}).\coqdoceol
\coqdocindent{2.00em}
\{\coqdoceol
\coqdocindent{3.00em}
\coqdoctac{apply} \coqdocvar{erase\_lbeta\_2313}.\coqdoceol
\coqdocindent{2.00em}
\}\coqdoceol
\coqdocindent{2.50em}
\coqdoctac{in} \coqdocvar{H2} \coqdockw{with} (\coqdocvar{pterm\_app} (\coqdocvar{pterm\_labs} \coqdocvar{t1}) \coqdocvar{u}) \coqdocvar{\_}.\coqdoceol
\coqdocemptyline
\coqdocemptyline
\coqdocindent{1.00em}
\coqdoctac{assert} (\coqdocvar{H2'} := \coqdocvar{H2}).\coqdoceol
\coqdocindent{1.00em}
\coqdoctac{assert} (\coqdockw{\ensuremath{\forall}} \coqdocvar{t'}, \coqdocvar{erase}(\coqdocvar{t'}) = \coqdocvar{t} \ensuremath{\rightarrow} \coqdockw{\ensuremath{\forall}} \coqdocvar{t2'}, \coqdocvar{erase}(\coqdocvar{t2'}) = \coqdocvar{t2} \ensuremath{\rightarrow} \coqdocvar{t'} -->>\coqdocvar{lB} \coqdocvar{t2'}).\coqdoceol
\coqdocindent{1.00em}
\{\coqdoceol
\coqdocindent{2.00em}
\coqdoctac{apply} \coqdocvar{erase\_lbeta}; \coqdoctac{assumption}.\coqdoceol
\coqdocindent{1.00em}
\}\coqdoceol
\coqdocindent{1.00em}
\coqdoctac{inversion} \coqdocvar{H1}; \coqdoctac{subst}.\coqdoceol
\coqdocindent{1.00em}
- \coqdoctac{inversion} \coqdocvar{H0}; \coqdoctac{subst}.\coqdoceol
\coqdocindent{2.00em}
\coqdoctac{assert} (\coqdocvar{erase} (\coqdocvar{pterm\_app} (\coqdocvar{pterm\_labs} \coqdocvar{t0}) \coqdocvar{u}) = \coqdocvar{pterm\_app} (\coqdocvar{pterm\_abs} \coqdocvar{t0}) \coqdocvar{u} \ensuremath{\rightarrow} \coqdockw{\ensuremath{\forall}} \coqdocvar{t2'}, \coqdocvar{erase} \coqdocvar{t2'} = \coqdocvar{t2} \ensuremath{\rightarrow} \coqdocvar{pterm\_app} (\coqdocvar{pterm\_labs} \coqdocvar{t0}) \coqdocvar{u} -->>\coqdocvar{lB} \coqdocvar{t2'}).\coqdoceol
\coqdocindent{2.00em}
\{\coqdoceol
\coqdocindent{3.00em}
\coqdoctac{apply} \coqdocvar{H}.\coqdoceol
\coqdocindent{2.00em}
\}\coqdoceol
\coqdocindent{2.00em}
\coqdoctac{clear} \coqdocvar{H}.\coqdoceol
\coqdocindent{2.00em}
\coqdoctac{assert} (\coqdocvar{erase} (\coqdocvar{pterm\_app} (\coqdocvar{pterm\_labs} \coqdocvar{t0}) \coqdocvar{u}) = \coqdocvar{pterm\_app} (\coqdocvar{pterm\_abs} \coqdocvar{t0}) \coqdocvar{u}).\coqdoceol
\coqdocindent{2.00em}
\{\coqdoceol
\coqdocindent{3.00em}
\coqdoctac{simpl}.\coqdoceol
\coqdocindent{3.00em}
\coqdoctac{rewrite} \coqdocvar{body\_erase}.\coqdoceol
\coqdocindent{3.00em}
- \coqdoctac{rewrite} \coqdocvar{term\_erase}.\coqdoceol
\coqdocindent{4.00em}
+ \coqdoctac{reflexivity}.\coqdoceol
\coqdocindent{4.00em}
+ \coqdoctac{assumption}.\coqdoceol
\coqdocindent{3.00em}
- \coqdoctac{assumption}.\coqdoceol
\coqdocindent{2.00em}
\}\coqdoceol
\coqdocindent{2.00em}
\coqdoctac{assert} (\coqdockw{\ensuremath{\forall}} \coqdocvar{t2'} : \coqdocvar{pterm}, \coqdocvar{erase} \coqdocvar{t2'} = \coqdocvar{t2} \ensuremath{\rightarrow} \coqdocvar{pterm\_app} (\coqdocvar{pterm\_labs} \coqdocvar{t0}) \coqdocvar{u} -->>\coqdocvar{lB} \coqdocvar{t2'}).\coqdoceol
\coqdocindent{2.00em}
\{\coqdoceol
\coqdocindent{3.00em}
\coqdoctac{apply} \coqdocvar{H5}.\coqdoceol
\coqdocindent{3.00em}
\coqdoctac{assumption}.\coqdoceol
\coqdocindent{2.00em}
\}\coqdoceol
\coqdocindent{2.00em}
\coqdoctac{clear} \coqdocvar{H} \coqdocvar{H5}.\coqdoceol
\coqdocindent{2.00em}
\coqdocvar{Admitted}.\coqdoceol
\coqdocemptyline
\end{coqdoccode}
