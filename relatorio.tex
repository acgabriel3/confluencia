\documentclass{article} 
\usepackage[utf8]{inputenc}
\usepackage{mathpartir}
\usepackage[all]{xy}
\usepackage[portuguese]{babel}
\usepackage[usenames]{color} 
\usepackage{amsmath}
\usepackage{amssymb}
\usepackage{pdfsync}
\usepackage{coqdoc}

\newcommand{\tto}{\twoheadrightarrow}

  \title{Formalizando uma prova da confluência do cálculo lambda em Coq}
  \author{Gabriel Alves Castro}
  \date{}

\begin{document}
\maketitle

\coqlibrary{confl lambda}{Library }{confl\_lambda}

\begin{coqdoccode}
\coqdocemptyline
\end{coqdoccode}
\section{Locally Nameless Notation}

\begin{coqdoccode}
\coqdocemptyline
\coqdocemptyline
\end{coqdoccode}
A notação de nomes locais é definida de forma a trabalhar com index's ligados à abstrações
para representar variáveis ligadas, e variáveis nomeadas para representar variáveis livres. Dessa
forma podemos ter um conjunto de símbolos nesta notação que não sejam válidos semanticamente. 
Vamos exemplificar: 

 Dada a abstração $\lambda.0$ sabemos que o index 0 representa a variável ligada à abstração
que apresentamos. O index 0 neste caso indica que existem 0 passos para em uma aplicação
substituir a variável ligada que o mesmo representa, estando esse index portanto ligado
diretamente à abstração apresentada, e sendo dessa maneira válido semanticamente. Porém,
sintaticamente o index pode ser um valor iteiro k qualquer. Assim, por exemplo, ao 
fazermos $\lambda.1$, estamos construindo semanticamente uma relação que não possui 
significado válido. Queremos representar variáveis ligadas por meio de index's e neste caso o 
index 1 não está ligado à nenhuma abstração, portanto não representa uma variável ligada
e não é válido para nossa representação.

 Assim podemos definir o conceito de pré-termo como sendo: Todo conjunto de símbolos sintaticamente
válidos, que ainda não temos certeza acerca da validade semântica. Matematicamente os pré-termos 
são definidos de acordo com a seguinte gramática: \begin{coqdoccode}
\coqdocemptyline
\coqdocnoindent
\coqdockw{Inductive} \coqdocvar{pterm} : \coqdockw{Set} :=\coqdoceol
\coqdocindent{1.00em}
\ensuremath{|} \coqdocvar{pterm\_bvar} : \coqdocvar{nat} \ensuremath{\rightarrow} \coqdocvar{pterm}\coqdoceol
\coqdocindent{1.00em}
\ensuremath{|} \coqdocvar{pterm\_fvar} : \coqdocvar{var} \ensuremath{\rightarrow} \coqdocvar{pterm}\coqdoceol
\coqdocindent{1.00em}
\ensuremath{|} \coqdocvar{pterm\_app}  : \coqdocvar{pterm} \ensuremath{\rightarrow} \coqdocvar{pterm} \ensuremath{\rightarrow} \coqdocvar{pterm}\coqdoceol
\coqdocindent{1.00em}
\ensuremath{|} \coqdocvar{pterm\_abs}  : \coqdocvar{pterm} \ensuremath{\rightarrow} \coqdocvar{pterm}\coqdoceol
\coqdocindent{1.00em}
\ensuremath{|} \coqdocvar{pterm\_labs}  : \coqdocvar{pterm} \ensuremath{\rightarrow} \coqdocvar{pterm}.\coqdoceol
\coqdocemptyline
\end{coqdoccode}
Um pré-termo é um termo do cálculo lambda que possui em si indexs de de Bruijin que não foram
substituídos. (esta definição está correta?)






Fixpoint fv (t : pterm) : vars :=
  match t with
  | pterm\_bvar i    => \{\}
  | pterm\_fvar x    => \footnote{\url{x}}
  | pterm\_app t1 t2 => (fv t1) \symbol{92}u (fv t2)
  | pterm\_abs t1    => (fv t1)
  | pterm\_labs t1    => (fv t1)
  end.


Fixpoint bv (t : pterm) : vars :=
  match t with
  | pterm\_bvar i    => \footnote{\url{i}}
  | pterm\_fvar x    => \{\}
  | pterm\_app t1 t2 => (bv t1) \symbol{92}u (bv t2)
  | pterm\_abs t1    => (bv t1)
  | pterm\_labs t1    => (bv t1)
  end.


Ltac gather\_vars\_with F :=
  let rec gather V :=
    match goal with
    | H: ?S |- \_ =>
      let FH := constr:(F H) in
      match V with
      | \{\} => gather FH
      | context \coqdocvar{FH} => fail 1
      | \_ => gather (FH \symbol{92}u V)
      end
    | \_ => V
    end in
  let L := gather \{\} in eval simpl in L.


Ltac gather\_vars :=
  let A := gather\_vars\_with (fun x : vars => x) in
  let B := gather\_vars\_with (fun x : var => \footnote{\url{ x }}) in
  let D := gather\_vars\_with (fun x : pterm => fv x) in
  constr:(A \symbol{92}u B \symbol{92}u D).


Ltac beautify\_fset V :=
  let rec go Acc E :=
     match E with
     | ?E1 \symbol{92}u ?E2 => let Acc1 := go Acc E1 in
                     go Acc1 E2
     | \{\}  => Acc
     | ?E1 => match Acc with
              | \{\} => E1
              | \_ => constr:(Acc \symbol{92}u E1)
              end
     end
  in go \{\} V.


Require Import List Omega.
Open Scope list\_scope.


Lemma max\_lt\_l :
  forall (x y z : nat), x <= y -> x <= max y z.
Proof.
  induction x; auto with arith.
  induction y; induction z; simpl; auto with arith.
Qed.


Lemma finite\_nat\_list\_max : forall (l : list nat),
  \{ n : nat | forall x, In x l -> x <= n \}.
Proof.
  induction l as  \ensuremath{|} \coqdocvar{l} \coqdocvar{ls} \coqdocvar{IHl} .

\begin{itemize}
\item  exists 0; intros x H; inversion H.

\item  inversion IHl as \coqdocvar{x} \coqdocvar{H}; clear IHl.
    exists (max x l).
    intros x' Hin.
    inversion Hin; subst.
    + auto with arith.
    + assert (x' <= x); auto using max\_lt\_l.

\end{itemize}
Qed.      


Lemma finite\_nat\_list\_max' : forall (l : list nat),
  \{ n : nat | \~{} In n l \}.
Proof.
  intros l. case (finite\_nat\_list\_max l); intros x H.
  exists (S x). intros J. assert (K := H \_ J); omega.
Qed.


Definition var\_gen (L : vars) : var :=
  proj1\_sig (finite\_nat\_list\_max' (elements L)).


Lemma var\_gen\_spec : forall E, (var\_gen E) \symbol{92}notin E.
Proof.
  unfold var\_gen. intros E.
  destruct (finite\_nat\_list\_max' (elements E)) as \coqdocvar{n} \coqdocvar{pf}.
  simpl. intros a. 
  destruct pf.
  apply elements\_spec1 in a.
  rewrite InA\_alt in a.
  destruct a as \coqdocvar{y} [\coqdocvar{H1} \coqdocvar{H2}].
  subst; assumption.
Qed.


Lemma var\_fresh : forall (L : vars), \{ x : var | x \symbol{92}notin L \}.
Proof.
  intros L. exists (var\_gen L). apply var\_gen\_spec.
Qed.


Ltac pick\_fresh\_gen L Y :=
  let Fr := fresh ``Fr'' in
  let L := beautify\_fset L in
  (destruct (var\_fresh L) as \coqdocvar{Y} \coqdocvar{Fr}).


Ltac pick\_fresh Y :=
  let L := gather\_vars in (pick\_fresh\_gen L Y).


Lemma notin\_union : forall x E F,
  x \symbol{92}notin (E \symbol{92}u F) <-> (x \symbol{92}notin E) /\symbol{92} (x \symbol{92}notin F).
Proof.
assert (not\_or: forall (A B: Prop), \~{}(A \symbol{92}/ B) <-> \~{} A /\symbol{92} \~{} B).
\{
  unfold not.
  split.

\begin{itemize}
\item  intro H.
    split.
    + intro H0.
      destruct H.
      left. 
      assumption.
    + intro H0.
      destruct H.
      right.
      assumption.

\item  intros H H0.
    destruct H.
    destruct H0; contradiction.

\end{itemize}
\}
intros x E F.
apply iff\_stepl with (\~{}((x \symbol{92}in E) \symbol{92}/ (x \symbol{92}in F))).

\begin{itemize}
\item  apply not\_or.

\item  split; unfold not; intros; destruct H; apply union\_spec in H0; assumption.

\end{itemize}
Qed.




Fixpoint open\_rec (k : nat) (u : pterm) (t : pterm) : pterm :=
  match t with
  | pterm\_bvar i    => if k === i then u else (pterm\_bvar i)
  | pterm\_fvar x    => pterm\_fvar x
  | pterm\_app t1 t2 => pterm\_app (open\_rec k u t1) (open\_rec k u t2)
  | pterm\_abs t1    => pterm\_abs (open\_rec (S k) u t1)
  | pterm\_labs t1    => pterm\_labs (open\_rec (S k) u t1)
  end.






Definition open t u := open\_rec 0 u t.






Notation ``\{ k \~{}> u \} t'' := (open\_rec k u t) (at level 67).






Notation ``t \^{}\^{} u'' := (open t u) (at level 67). 




Notation ``t \^{} x'' := (open t (pterm\_fvar x)).   


Fixpoint close\_rec  (k : nat) (x : var) (t : pterm) : pterm :=
  match t with
  | pterm\_bvar i    => pterm\_bvar i
  | pterm\_fvar x'    => if x' == x then (pterm\_bvar k) else pterm\_fvar x'
  | pterm\_app t1 t2 => pterm\_app (close\_rec k x t1) (close\_rec k x t2)
  | pterm\_abs t1    => pterm\_abs (close\_rec (S k) x t1)
  | pterm\_labs t1 => pterm\_labs (close\_rec (S k) x t1)
  end.


Definition close t x := close\_rec 0 x t.




Inductive term : pterm -> Prop :=
  | term\_var : forall x,
      term (pterm\_fvar x)
  | term\_app : forall t1 t2,
      term t1 -> 
      term t2 -> 
      term (pterm\_app t1 t2)
  | term\_abs : forall L t1,
      (forall x, x \symbol{92}notin L -> term (t1 \^{} x)) ->
      term (pterm\_abs t1).






Definition body t :=
  exists L, forall x, x \symbol{92}notin L -> term (t \^{} x).








Inductive lterm : pterm -> Prop :=
  | lterm\_var : forall x,
      lterm (pterm\_fvar x)
  | lterm\_app : forall t1 t2,
      lterm t1 -> 
      lterm t2 -> 
      lterm (pterm\_app t1 t2)
  | lterm\_abs : forall L t1,
      (forall x, x \symbol{92}notin L -> lterm (t1 \^{} x)) ->
      lterm (pterm\_abs t1)
  | lterm\_labs : forall L t1 t2,
      (forall x, x \symbol{92}notin L -> lterm (t1 \^{} x)) ->
      lterm t2 ->
      lterm (pterm\_app (pterm\_labs t1) t2).




Definition lbody t :=
  exists L, forall x, x \symbol{92}notin L -> lterm (t \^{} x).


Hint Constructors lterm term.










Fixpoint pterm\_size (t : pterm) : nat :=
 match t with
 | pterm\_bvar i    => 1
 | pterm\_fvar x    => 1
 | pterm\_app t1 t2 => (pterm\_size t1) + (pterm\_size t2)
 | pterm\_abs t1    => 1 + (pterm\_size t1)
 | pterm\_labs t1   => 1 + (pterm\_size t1)
 end.


Lemma pterm\_size\_gt\_0: forall t, pterm\_size t > 0.
Proof.
  intro t; induction t.

\begin{itemize}
\item  simpl.
    auto.

\item  simpl.
    auto.

\item  simpl.
    apply Nat.add\_pos\_r; assumption.

\item  simpl.
    auto.

\item  simpl.
    auto.

\end{itemize}
Qed.


Lemma pterm\_size\_open\_rec: forall t n x, pterm\_size t = pterm\_size (open\_rec n (pterm\_fvar x) t).
Proof.
  intro t; induction t.

\begin{itemize}
\item  intros n0 x.
    simpl.
    destruct(n0 === n); reflexivity.

\item  intros n x.
    reflexivity.

\item  intros n x.
    simpl.
    rewrite <- IHt1.
    rewrite <- IHt2.
    reflexivity.

\item  intros n x.
    simpl.
    rewrite <- IHt.
    reflexivity.

\item  intros n x.
    simpl.
    rewrite <- IHt.
    reflexivity.

\end{itemize}
Qed.




Lemma strong\_induction :
 forall (P: nat -> Prop),
   (forall n, (forall m, m < n -> P m) -> P n) ->
   (forall n, P n).
Proof.
  intros P H.
  cut (forall n m, m < n -> P m).

\begin{itemize}
\item  intros H' n.
    apply H.
    apply H'.

\item  induction n.
    + intros m H'.
      inversion H'.
    + intros m Hlt.
      apply H.
      intros m' Htl.
      apply IHn.
      apply lt\_n\_Sm\_le in Hlt.
      apply Nat.lt\_le\_trans with m; assumption.

\end{itemize}
Qed.


Lemma pterm\_size\_induction :
 forall P : pterm -> Prop,
 (forall n, P (pterm\_bvar n)) ->
 (forall x, P (pterm\_fvar x)) ->
 (forall t1 t2, P t1 -> P t2 -> P (pterm\_app t1 t2)) ->
 (forall t1,
    (forall t2 x, x \symbol{92}notin fv t2 -> pterm\_size t2 = pterm\_size t1 ->
    P (t2 \^{} x)) -> P (pterm\_abs t1)) ->
 (forall t1,
    (forall t2 x, x \symbol{92}notin fv t2 -> pterm\_size t2 = pterm\_size t1 ->
    P (t2 \^{} x)) -> P (pterm\_labs t1)) ->
 (forall t, P t).
Proof.
  intros P h1 h2 h3 h4 h5 t.
  remember (pterm\_size t) as n.
  generalize dependent t.
  induction n using strong\_induction.
  intro t; destruct t.

\begin{itemize}
\item  intro Hsize.
    apply h1.

\item  intro Hsize.
    apply h2.

\item  simpl.
    intro Hsize.
    apply h3.
    + apply H with (pterm\_size t1).
      * rewrite Hsize.
        assert (Ht2: pterm\_size t2 > 0).
        \{
          apply pterm\_size\_gt\_0.
        \}
        rewrite Nat.add\_comm.
        apply Nat.lt\_add\_pos\_l.
        assumption.
      * reflexivity.
    + apply H with (pterm\_size t2).
      * rewrite Hsize.
        assert (Ht2: pterm\_size t1 > 0).
        \{
          apply pterm\_size\_gt\_0.
        \}
        apply Nat.lt\_add\_pos\_l.
        apply Ht2.
      * reflexivity.

\item  simpl.
    intro Hsize.
    apply h4.
    intros t2 x Hfv Hequals.
    apply H with (pterm\_size t0).
    rewrite Hsize.
    apply Nat.lt\_succ\_diag\_r.
    assert (Hopen: pterm\_size t2 = pterm\_size (open\_rec 0 (pterm\_fvar x) t2)).
    \{
      apply pterm\_size\_open\_rec.
    \}
    unfold open.
    rewrite <- Hopen.
    symmetry; assumption.

\item  simpl.
    intro Hsize.
    apply h5.
    intros t2 x Hfv Hequals.
    apply H with (pterm\_size t0).
    rewrite Hsize.
    apply Nat.lt\_succ\_diag\_r.
    assert (Hopen: pterm\_size t2 = pterm\_size (open\_rec 0 (pterm\_fvar x) t2)).
    \{
      apply pterm\_size\_open\_rec.
    \}
    unfold open.
    rewrite <- Hopen.
    symmetry; assumption.

\end{itemize}
Qed.


Fixpoint lc\_at (k:nat) (t:pterm) : Prop :=
  match t with
  | pterm\_bvar i    => i < k
  | pterm\_fvar x    => True
  | pterm\_app t1 t2 => lc\_at k t1 /\symbol{92} lc\_at k t2
  | pterm\_abs t1    => lc\_at (S k) t1
  | pterm\_labs t1    => lc\_at (S k) t1
  end.


Lemma lc\_at\_open\_rec\_rename: forall t x y m n, lc\_at m (open\_rec n (pterm\_fvar x) t) -> lc\_at m (open\_rec n (pterm\_fvar y) t).
Proof.
  intro t; induction t.

\begin{itemize}
\item  intros x y m n0.
    simpl. destruct (n0 === n); subst.
    + intro H.
      auto.
    + intro H; assumption.

\item  intros x y m n H.
    auto.

\item  intros x y m n. 
    simpl.
    intro H.
    destruct H as \coqdocvar{H1} \coqdocvar{H2}.
    split.
    + apply IHt1 with x; assumption.
    + apply IHt2 with x; assumption.

\item  intros x y m n.
    simpl.
    intro H.
    apply IHt with x; assumption.

\item  intros x y m n.
    simpl.
    intro H.
    apply IHt with x; assumption.

\end{itemize}
Qed.


Lemma lc\_at\_weaken: forall t n m, n <= m -> lc\_at n t -> lc\_at m t.
Proof.
  intro t; induction t.

\begin{itemize}
\item  intros n' m Hleq.
    simpl.
    intro Hlt.
    apply Nat.lt\_le\_trans with n'; assumption.

\item  intros n m Hleq.
    auto.

\item  intros n m Hleq.
    simpl.
    intro H.
    destruct H as \coqdocvar{H1} \coqdocvar{H2}.
    split.
    + apply IHt1 with n; assumption.
    + apply IHt2 with n; assumption.

\item  intros n m Hleq.
    simpl.
    apply IHt.
    apply le\_n\_S; assumption.

\item  intros n m Hleq.
    simpl.
    intro H.
    apply IHt with (S n).
    + apply le\_n\_S; assumption.
    + assumption.

\end{itemize}
Qed.


Lemma lc\_at\_open: forall t m x, lc\_at m (\{m \~{}> pterm\_fvar x\} t) <-> lc\_at (S m) t.
Proof.
  intros t m x; split.

\begin{itemize}
\item  generalize dependent x.
    generalize dependent m.
    induction t.
    + intros m x.
      simpl.
      destruct (m === n).
      * subst.
        intro H.
        apply Nat.lt\_succ\_diag\_r.
      * simpl.
        intro H.
        apply Nat.lt\_lt\_succ\_r; assumption.
    + intros m x.
      auto.
    + intros m x.
      simpl.
      intro H.
      destruct H as \coqdocvar{H1} \coqdocvar{H2}.
      split.
      * apply IHt1 with x; assumption.
      * apply IHt2 with x; assumption.
    + intros m x.
      simpl.
      apply IHt.
    + intros m x.
      simpl.
      intro H.
      apply IHt with x; assumption.

\item  generalize dependent x.
    generalize dependent m.
    induction t.
    + intros m x Hat.
      simpl.
      destruct(m ===n).
      * simpl.
        auto.
      * unfold lc\_at in *.
        apply lt\_n\_Sm\_le in Hat.
        apply le\_lt\_or\_eq in Hat.
        destruct Hat.
        ** assumption.
        ** symmetry in H.
          contradiction.
    + intros m x H.
      simpl in *.
      auto.
    + intros m x H.
      simpl in H.
      destruct H as \coqdocvar{H1} \coqdocvar{H2}.
      simpl; split.
      * apply IHt1; assumption.
      * apply IHt2; assumption.
    + intros m x H.
      simpl in H.
      simpl.
      apply IHt; assumption.
    + intros m x H.
      simpl in H.
      simpl.
      apply IHt; assumption.

\end{itemize}
Qed.


Lemma term\_to\_lc\_at : forall t, term t -> lc\_at 0 t.
Proof.
  intros t Hterm.
  induction Hterm.

\begin{itemize}
\item  simpl.
    auto.

\item  simpl.
    split; assumption.

\item  pick\_fresh x.
    apply notin\_union in Fr.
    destruct Fr as \coqdocvar{Fr} \coqdocvar{H1}.
    apply H0 in Fr.
    unfold open in Fr.
    apply lc\_at\_open in Fr.
    assumption.

\end{itemize}
Qed.




Lemma lterm\_to\_lc\_at : forall t, lterm t -> lc\_at 0 t.
Proof.
  intros t Hterm.
  induction Hterm.

\begin{itemize}
\item  simpl.
    auto.

\item  simpl; split; assumption.

\item  simpl.
    pick\_fresh x.
    apply notin\_union in Fr.
    destruct Fr as \coqdocvar{Fr} \coqdocvar{H1}.
    apply H0 in Fr.
    apply lc\_at\_open in Fr.
    assumption.

\item  simpl; split.
    + pick\_fresh x.
    apply notin\_union in Fr.
    destruct Fr as \coqdocvar{Fr} \coqdocvar{H2}.
    apply notin\_union in Fr.
    destruct Fr as \coqdocvar{Fr} \coqdocvar{H1}.
    apply H0 in Fr.
    apply lc\_at\_open in Fr.
    assumption.
    + assumption.

\end{itemize}
Qed.




Fixpoint has\_free\_index (k:nat) (t:pterm) : Prop :=
  match t with
    | pterm\_bvar n => if (k === n) then True else False
    | pterm\_fvar x => False
    | pterm\_app t1 t2 => (has\_free\_index k t1) \symbol{92}/ (has\_free\_index k t2)
    | pterm\_abs t1 => has\_free\_index (S k) t1
    | pterm\_labs t1 => has\_free\_index (S k) t1
  end.


Lemma deMorgan: forall p q, (\~{} p) /\symbol{92} (\~{} q) -> \~{}(p \symbol{92}/ q).
Proof.
  intros p q H1.
  intros or.
  destruct(or).

\begin{itemize}
\item  destruct(H1).
    contradiction.

\item  destruct(H1).
    contradiction.

\end{itemize}
Qed.


Lemma term\_rename: forall t x y, term (t \^{} x) -> term (t \^{} y).
Proof.
  intros t x y Hterm.
  apply term\_to\_lc\_at in Hterm.
  Admitted.




Lemma ind\_max: forall t u n, term (\{n \~{}> u\}t) -> \~{} (has\_free\_index (S n) t).
Proof.
Admitted.


Lemma body\_not\_S: forall t n, body t -> not (has\_free\_index (S n) t).
Proof.
  intros t0 n H1; induction t0.

\begin{itemize}
\item  unfold body in H1.
    unfold has\_free\_index.
    destruct(S n === n0); subst.
    + destruct H1 as \coqdocvar{L}.
      pick\_fresh x.
      apply notin\_union in Fr.
      destruct Fr as \coqdocvar{Fr} \coqdocvar{Hn}.
      apply H in Fr.
      unfold open in Fr.
      simpl in Fr.
      inversion Fr.
    + intros Htrue.
      assumption.

\item  unfold body in H1.
    unfold has\_free\_index.
    intros HFalse.
    assumption.

\item  simpl.
    unfold body in H1.
    destruct H1.
    pick\_fresh y.
     apply notin\_union in Fr.
     destruct Fr as \coqdocvar{Fr} \coqdocvar{H2}.
     apply notin\_union in Fr.
     destruct Fr as \coqdocvar{Fr} \coqdocvar{H1}.
     apply notin\_union in Fr.
     destruct Fr as \coqdocvar{Fr} \coqdocvar{Hn}.
     apply H in Fr.
     change (pterm\_app t0\_1 t0\_2 \^{} y) with (pterm\_app (t0\_1 \^{} y) (t0\_2 \^{} y)) in Fr.
     inversion Fr; subst.
     apply deMorgan.
     split.
    + apply IHt0\_1.
      unfold body.
      exists (fv t0\_1).
      intros x0 Hfv.
      apply term\_rename with y.
      assumption.
    + admit.

\item  intro Hfree.
    clear IHt0.
    unfold body in H1.
    destruct H1 as \coqdocvar{L}.
    pick\_fresh x.
    apply notin\_union in Fr.
    destruct Fr as \coqdocvar{Fr} \coqdocvar{H0}.
    apply notin\_union in Fr.
    destruct Fr as \coqdocvar{Fr} \coqdocvar{Hn}.
    apply H in Fr.
    unfold open in Fr.
    simpl in Fr.
    inversion Fr; subst.
    pick\_fresh z.
    apply notin\_union in Fr0.
    destruct Fr0 as \coqdocvar{Fr0} \coqdocvar{Ht0}.
    apply notin\_union in Fr0.
    destruct Fr0 as \coqdocvar{Fr0} \coqdocvar{Hx}.
    apply notin\_union in Fr0.
    destruct Fr0 as \coqdocvar{Fr0} \coqdocvar{Hn'}.
    apply notin\_union in Fr0.
    destruct Fr0 as \coqdocvar{Fr0} \coqdocvar{HL0}.
    apply H2 in HL0.
    unfold open in HL0.
    simpl in Hfree.
    clear L H Fr Hn H0 L0 H2 Fr0 Hn' Hx Ht0.
    replace (\{0 \~{}> pterm\_fvar z\} (\{1 \~{}> pterm\_fvar x\} t0)) with (\{1 \~{}> pterm\_fvar x\} (\{0 \~{}> pterm\_fvar z\} t0)) in HL0.
    + apply ind\_max in HL0.

\end{itemize}
Admitted.


Lemma open\_rec\_close\_rec\_term: forall t u k, \~{}(has\_free\_index k t) -> open\_rec k u t = t.
Proof.
  intro t; induction t.

\begin{itemize}
\item  intros u k H1.
    simpl.
    destruct(k === n).
    + rewrite <- e.
      admit.
    + reflexivity.

\item  intros u k H1.
    simpl.
    reflexivity.

\item  intros u k H1.

\end{itemize}
Admitted.


Lemma subst\_body: forall t u n, body t -> \{S n \~{}> u\} t = t.
Proof.
  intros t u n Hbody.
  apply open\_rec\_close\_rec\_term.
  apply body\_not\_S; assumption.
Qed.  
  

Lemma subst\_term: forall t u n, term t -> \{n \~{}> u\} t = t.
Proof.
  intros t u n H.
  generalize dependent n.
  generalize dependent u.
  induction H.

\begin{itemize}
\item  intros u n.
    simpl.
    reflexivity.

\item  intros u n.
    simpl.
    rewrite IHterm1.
    rewrite IHterm2.
    reflexivity.

\item  intros u n.
    simpl.
    rewrite subst\_body.
    + reflexivity.
    + unfold body.
      exists L.
      assumption.

\end{itemize}
Qed.


Lemma subst\_lterm: forall t u n, lterm t -> \{n \~{}> u\} t = t.
Proof.
  intros t u n H.
  generalize dependent n.
  generalize dependent u.
  induction H.

\begin{itemize}
\item  admit.

\item  admit.

\item  admit.

\item  intros u n.
    simpl.
    f\_equal.

\end{itemize}
Admitted.


Lemma abs\_body: forall t1 t2 L, (forall x, x \symbol{92}notin L -> t1\^{}x = t2\^{}x) -> pterm\_abs t1 = pterm\_abs t2.
Proof.
  intro t1; induction t1.

\begin{itemize}
\item  intro t2; induction t2.
    + intros L H.
      unfold open in H.
      Admitted.

\end{itemize}


Lemma subst\_open: forall t u x n,  (\{S n \~{}> u\} t) \^{} x = \{S n \~{}> (u \^{} x)\} (t \^{} x). 
Proof.
  Admitted.


Lemma subst\_term': forall t, (forall u n, term t -> \{n \~{}> u\} t = t).
Proof.
  assert (Hind := term\_ind (fun t => forall u n, term t -> \{n \~{}> u\} t = t)).
  intro t; apply Hind.

\begin{itemize}
\item  intros x u n Hterm.
    reflexivity.

\item  intros t1 t2 Ht1 IHt1 Ht2 IHt2 u n Hterm.
    simpl.
    inversion Hterm; subst.
    rewrite IHt1.
    + rewrite IHt2.
      * reflexivity.
      * assumption.
    + assumption.

\item  intros L t1 Ht1 IHt1 u n Hterm.
    inversion Hterm; subst.
    simpl.
    apply abs\_body with L0.
    intros x H.
    apply H0 in H.
    generalize dependent H.
    rewrite subst\_open.
    assert (IHt1' := (IHt1 x)).
    apply IHt1.

\end{itemize}
Admitted.


Lemma subst\_lemma\_lterms: forall (t1 t2 t3: pterm) (i j:nat), lterm t2 -> lterm t3 -> i <> j -> \{j \~{}> t3\} (\{i \~{}> t2\} t1) = \{i \~{}> t2\} (\{j \~{}> t3\} t1).
Proof.
  intro t1; induction t1.

\begin{itemize}
\item  intros t2 t3 i j H2 H3 Hle.
    simpl.
    destruct (i ===n); subst.
    + destruct (j === n).
      * symmetry in e.
        contradiction.
      * Admitted.

\end{itemize}








Definition Rel (A:Type) := A -> A -> Prop.


Inductive contextual\_closure (R: Rel pterm) : Rel pterm :=
  | redex : forall t s, R t s -> contextual\_closure R t s
  | app\_left : forall t t' u, contextual\_closure R t t' -> term u ->
	  		      contextual\_closure R (pterm\_app t u) (pterm\_app t' u)
  | app\_right : forall t u u', contextual\_closure R u u' -> term t ->
	  		       contextual\_closure R (pterm\_app t u) (pterm\_app t u')
  | abs\_in : forall t t' L, (forall x, x \symbol{92}notin L -> contextual\_closure R (t\^{}x) (t'\^{}x)) ->
                               contextual\_closure R (pterm\_abs t) (pterm\_abs t').


Inductive lcontextual\_closure (R: Rel pterm) : Rel pterm :=
  | lredex : forall t s, R t s -> lcontextual\_closure R t s
  | lapp\_left : forall t t' u, lcontextual\_closure R t t' -> lterm u ->
	  		      lcontextual\_closure R (pterm\_app t u) (pterm\_app t' u)
  | lapp\_right : forall t u u', lcontextual\_closure R u u' -> lterm t ->
	  		       lcontextual\_closure R (pterm\_app t u) (pterm\_app t u')
  | labs\_in : forall t t' L, (forall x, x \symbol{92}notin L -> contextual\_closure R (t\^{}x) (t'\^{}x)) ->
                               lcontextual\_closure R (pterm\_abs t) (pterm\_abs t')
  | l\_abs\_in : forall t t' L, (forall x, x \symbol{92}notin L -> lcontextual\_closure R (t\^{}x) (t'\^{}x)) ->
                               lcontextual\_closure R (pterm\_labs t) (pterm\_labs t').


Fixpoint erase (t:pterm) : pterm :=
  match t with
  | pterm\_app t1 t2 => pterm\_app (erase t1) (erase t2)
  | pterm\_abs t1 => pterm\_abs (erase t1)
  | pterm\_labs t1 => pterm\_abs (erase t1)
  | \_ => t
  end.




Lemma erase\_idemp: forall a, erase (erase a) = erase a.
Proof.
  induction a.

\begin{itemize}
\item  reflexivity.

\item  reflexivity.

\item  simpl.
    rewrite IHa1.
    rewrite IHa2.
    reflexivity.

\item  simpl.
    rewrite IHa; reflexivity.

\item  simpl.
    rewrite IHa; reflexivity.

\end{itemize}
Qed.


Inductive refltrans (red: Rel pterm) : Rel pterm :=
| reflex: forall a, refltrans red a a
| atleast1: forall a b, red a b -> refltrans red a b
| rtrans: forall a b c, refltrans red a b -> refltrans red b c -> refltrans red a c.


Inductive refltrans' (red: Rel pterm) : Rel pterm :=
| refl: forall a, refltrans' red a a
| rtrans': forall a b c, red a b -> refltrans' red b c -> refltrans' red a c.


Lemma refltrans\_equiv: forall (R: Rel pterm) (a b: pterm), refltrans R a b <-> refltrans' R a b.
Proof.
  intros R a b; split.

\begin{itemize}
\item  intro H.
    induction H.
    + apply refl.
    + apply rtrans' with b.
      * assumption.
      * apply refl.
    + clear H H0.
      induction IHrefltrans1.
      * assumption.
      * apply rtrans' with b.
        ** assumption.
        ** apply IHIHrefltrans1; assumption.

\item  intro H.
    induction H.
    + apply reflex.
    + apply rtrans with b.
      * apply atleast1; assumption.
      * assumption.

\end{itemize}
Qed.    


Inductive rule\_b : Rel pterm  :=
  reg\_rule\_b : forall (t u:pterm), body t -> term u ->
    rule\_b (pterm\_app(pterm\_abs t) u) (t \^{}\^{} u).
Notation ``t -->B u'' := (contextual\_closure rule\_b t u) (at level 60).
Notation ``t -->>B u'' := (refltrans (contextual\_closure rule\_b) t u) (at level 60).


Inductive rule\_lb : Rel pterm  :=
  | reg\_rule\_bb : forall (t u:pterm), body t -> term u ->
    rule\_lb (pterm\_app(pterm\_abs t) u) (t \^{}\^{} u)
  | reg\_rule\_lb : forall (t u:pterm),
    body t -> term u ->
    rule\_lb (pterm\_app(pterm\_labs t) u) (t \^{}\^{} u).
Notation ``t -->lB u'' := (lcontextual\_closure rule\_lb t u) (at level 60).
Notation ``t -->>lB u'' := (refltrans (lcontextual\_closure rule\_lb) t u) (at level 60).


Fixpoint phi (t:pterm) : pterm :=
  match t with
  | pterm\_app t1 t2 => match t1 with
                      | pterm\_labs t1' => (phi t1') \^{}\^{} (phi t2)
                      | \_ => pterm\_app (phi t1) (phi t2)
                      end
  | pterm\_abs t1 => pterm\_abs (phi t1)
  | pterm\_labs t1 => pterm\_labs (phi t1)
  | \_ => t
  end.


Lemma phi\_open\_rec\_fvar: forall t n x, phi(open\_rec n (pterm\_fvar x) t) = open\_rec n (pterm\_fvar x) (phi t).
Proof.  
  intro t; induction t.

\begin{itemize}
\item  intros n' x.
    simpl.
    destruct(n' === n).
    + reflexivity.
    + reflexivity.

\item  intros n x.
    reflexivity.

\item  intros n x.
    change (\{n \~{}> (pterm\_fvar x)\} pterm\_app t1 t2) with (pterm\_app (\{n \~{}> (pterm\_fvar x)\} t1) (\{n \~{}> (pterm\_fvar x)\} t2)).
    generalize dependent IHt1.
    case t1.
    + intros n0 IHt1.
      simpl (phi (pterm\_app (pterm\_bvar n0) t2)).
      simpl ( \{n \~{}> pterm\_fvar x\} pterm\_app (pterm\_bvar n0) (phi t2)).
      simpl (\{n \~{}> pterm\_fvar x\} pterm\_bvar n0).
      destruct(n === n0); subst.
      * simpl.
        rewrite IHt2.
        reflexivity.
      * simpl.
        rewrite IHt2.
        reflexivity.
    + intros v IHt1.
      simpl.
      f\_equal.
      apply IHt2. 
    + intros t11 t12 IHt1.
      simpl (\{n \~{}> pterm\_fvar x\} pterm\_app t11 t12).
      change (phi
    (pterm\_app (pterm\_app (\{n \~{}> pterm\_fvar x\} t11) (\{n \~{}> pterm\_fvar x\} t12))
               (\{n \~{}> pterm\_fvar x\} t2))) with
    (pterm\_app (phi (pterm\_app (\{n \~{}> pterm\_fvar x\} t11) (\{n \~{}> pterm\_fvar x\} t12)))
               (phi (\{n \~{}> pterm\_fvar x\} t2))).
      change (phi (pterm\_app (pterm\_app t11 t12) t2)) with
          (pterm\_app (phi (pterm\_app t11 t12)) (phi t2)).
      change (\{n \~{}> pterm\_fvar x\} pterm\_app (phi (pterm\_app t11 t12)) (phi t2)) with (pterm\_app (\{n \~{}> pterm\_fvar x\}(phi (pterm\_app t11 t12))) (\{n \~{}> pterm\_fvar x\}(phi t2))).
      rewrite IHt2.
      rewrite <- IHt1.
      reflexivity.
    + intros t1' IHt1.
      simpl in *.
      rewrite IHt2.
      rewrite <- IHt1.
      reflexivity.
    + intros t1' IHt1.


      simpl (\{n \~{}> pterm\_fvar x\} pterm\_labs t1').
      simpl in *.
      rewrite IHt2.
      replace (\{n \~{}> pterm\_fvar x\} (phi t1' \^{}\^{} phi t2)) with (phi (\{S n \~{}> pterm\_fvar x\} t1') \^{}\^{} (\{n \~{}> pterm\_fvar x\} phi t2)).
      * reflexivity.
      * assert (IHt1' := IHt1 n x).
        clear IHt1.
        inversion IHt1'. clear IHt1'.
        rewrite H0.
        Admitted.

\end{itemize}


Corollary phi\_open: forall t x, phi(t\^{}x) = (phi t)\^{}x.
Proof.
  intros t x.
  unfold open.
  Admitted.
  

Lemma phi\_term: forall t, lterm t -> term (phi t).
Proof.
  intros t Hlterm.
  induction Hlterm.

\begin{itemize}
\item  simpl.
    apply term\_var. 

\item  generalize dependent t2.
    induction t1.
    + simpl in *.
      inversion Hlterm1.
    + simpl in *.
      intros t2 Hlterm2 Hterm2.
      apply term\_app.
        * apply term\_var.
        * assumption. 
    + intros t2 Hlterm2 Hterm2.
      admit.
    + intros t2 Hlterm2 Hterm2.
      simpl.
      apply term\_app.
        * simpl in IHHlterm1.
          assumption.
        * assumption. 
    + intros t2 Hlterm2 Hterm2.
      clear IHt1.
      simpl in *.
      inversion Hlterm1.

\item  simpl.    
    unfold open.
    admit.

\item  simpl.

\end{itemize}
Admitted.


Lemma term\_fvar\_to\_term: forall t1 t2 t3 x, term (phi (pterm\_app t1 t2)\^{}x) -> term t3 -> term (phi (pterm\_app t1 t2)\^{}\^{}t3). 
Proof.
Admitted.  


Lemma term\_phi\_open: forall t1 t2 x L,  x \symbol{92}notin L -> term (phi (t1 \^{} x)) -> term (phi t2) -> term (phi t1 \^{}\^{} phi t2).
Proof.
  intro t; induction t.

\begin{itemize}
\item  intros t2 x L HL Hterm1 Hterm2.
    unfold open in *.
    generalize dependent n.
    intro n; case n.
    + intro H.
      assumption.
    + intros n'; simpl.
      intro H; inversion H.

\item  intros t3 x L HL.
    intros Hfvphi Ht3.
    simpl.
    assumption.

\item  intros t2' x L HL Hterm1 Hterm2.
    unfold open in *.
   
\end{itemize}
Admitted.
  



Lemma lterm\_preserves\_fvar : forall M x, erase M = (pterm\_fvar x) -> M = (pterm\_fvar x).
Proof.
  induction M.

\begin{itemize}
\item  intros x H.
    simpl in H.
    exact H.

\item  intros x H.
    exact H.

\item  intros x H.
    simpl in H.
    inversion H.

\item  intros x H.
    simpl in H.
    inversion H.

\item  intros x H.
    simpl in H.
    inversion H.

\end{itemize}
Qed.


Lemma lterm\_preserves\_bvar : forall M n, erase M = (pterm\_bvar n) -> M = (pterm\_bvar n).
Proof.
  Admitted.


Lemma lterm\_preserves\_app : forall M N L, erase M = (pterm\_app N L) -> exists N' L', M = (pterm\_app N' L').
Proof.
  Admitted.


Lemma lterm\_preserves\_abs : forall M N, erase M = (pterm\_abs N) -> exists N', M = (pterm\_abs N') \symbol{92}/ M = (pterm\_labs N').
Proof.
Admitted.


Lemma open\_rec\_preserves\_labs: forall t u k, (open\_rec k u (pterm\_labs t)) = (pterm\_labs (open\_rec (S k) u t)).
Proof.
  intros t u k .
  reflexivity.
Qed.


Lemma erase\_open\_rec : forall (M N: pterm) (k : nat), erase (\{k \~{}> N\} M) = \{k \~{}> (erase N)\} (erase M).
Proof.
  induction M.

\begin{itemize}
\item  intros N K.
    simpl.
    destruct (K === n).
    + reflexivity.
    + reflexivity.

\item  intros N k. 
    reflexivity.

\item  intros N k.
    simpl.
    f\_equal.
    + apply IHM1.
    + apply IHM2.

\item  intros N k.
    simpl.
    f\_equal.
    apply IHM.

\item  intros N k.
    simpl.
    f\_equal.
    apply IHM.

\end{itemize}
Qed.


Corollary erase\_open : forall M N: pterm, erase (M \^{}\^{} N) = (erase M) \^{}\^{} (erase N).
Proof.
  unfold open.
  intros M N.
  apply  erase\_open\_rec.
Qed.


Lemma phi\_subst\_rec: forall (M N: pterm) (k: nat), term N -> phi (\{k \~{}> N\} M) = \{k \~{}> (phi N)\}(phi M).
Proof.
  induction M.

\begin{itemize}
\item  intros N k.
    simpl.
    destruct (k === n).
    + reflexivity.
    + reflexivity.

\item  intros N k.
    simpl.
    reflexivity.

\item  intros N k.
    generalize dependent M1. 
    intro M1.
    case M1.
    + intros M1' IHM1.
      change (phi (pterm\_app (pterm\_bvar M1') M2)) with  (pterm\_app (phi(pterm\_bvar M1')) (phi M2)).
      change ( \{k \~{}> phi N\} pterm\_app (phi (pterm\_bvar M1')) (phi M2)) with
          ( pterm\_app ( \{k \~{}> phi N\}(phi (pterm\_bvar M1'))) ( \{k \~{}> phi N\}(phi M2))).
      assert( IH' := IHM1 N k).
      change (\{k \~{}> N\} pterm\_app (pterm\_bvar M1') M2) with (pterm\_app (\{k \~{}> N\}(pterm\_bvar M1')) (\{k \~{}> N\}M2)).
      admit.     + intros N0 k0.
      simpl in *.
      f\_equal.

\end{itemize}
                                              

                  Admitted.    


Corollary phi\_subst: forall M N, phi (M \^{}\^{} N) = (phi M) \^{}\^{} (phi N). 
Proof.
  Admitted.


Lemma phi\_prop: forall M N : pterm, lterm M -> lterm N -> (M -->>lB N) -> (phi M) -->>B (phi N).
Proof.
  intros M N Hterm1 Hterm2 H.
  induction H.

\begin{itemize}
\item  Admitted.

\end{itemize}






Lemma erase\_prop\_str: forall M' M N , pterm\_app (pterm\_abs M) N = erase M' -> exists u v, erase u = M -> erase v = N -> (M' = pterm\_app (pterm\_labs u) v) \symbol{92}/ (M' = pterm\_app (pterm\_abs u) v).
Proof.
Admitted.


Lemma erase\_prop1 : forall M N: pterm, term M -> term N -> (M -->B N) -> forall M' N', (erase M' = M) /\symbol{92} (erase N' = N) ->  (M' -->lB N').
Proof.
  intros M N HtM HtN Hred.


  induction Hred.

\begin{itemize}
\item  inversion H; subst.
       Admitted.
   
\end{itemize}


Lemma erase\_prop : forall M N M' N': pterm, term M -> term N -> (M -->>B N) -> erase M' = M -> erase N' = N ->  (M' -->>lB N').
Proof.
  intros M N M' N' HtM HtN Hred HeM HeN.
  induction Hred.

\begin{itemize}
\item  Admitted.

\end{itemize}




Lemma phi\_preserves\_term: forall t, term t -> term (phi t).
Proof.
  intros t H.
  induction H.
  Admitted.


Lemma beta\_phi\_one\_step: forall t1 t2, t1 -->lB t2 -> phi(t1) -->B phi(t2).
Proof.
  intros t1 t2 H.
  induction H.

\begin{itemize}
\item  inversion H; subst.
    + simpl.
      rewrite phi\_subst.
      apply redex.
      apply reg\_rule\_b.
      * admit.        
      * apply phi\_preserves\_term; assumption.
    + Admitted.

\end{itemize}


Lemma beta\_phi: forall t1 t2, t1 -->>lB t2 -> phi(t1) -->>B phi(t2).
Proof.
  intros t1 t2 H.
  induction H.

\begin{itemize}
\item  apply reflex.

\item  apply beta\_phi\_one\_step in H.
    apply atleast1; assumption.

\item  Admitted.

\end{itemize}


Lemma erase\_phi: forall t t1 t2, erase(t) = t1 -> phi(t) = t2 -> t1 -->>B t2.
Proof.
  

  intro t; induction t using pterm\_size\_induction.

\begin{itemize}
\item  intros t1 t2 Herase Hphi.
    simpl in *.
    rewrite Herase in Hphi.
    rewrite Hphi.
    apply reflex.

\item  intros t1 t2 Herase Hphi.
    simpl in *.
    rewrite Herase in Hphi.
    rewrite Hphi.
    apply reflex.

\item  

\end{itemize}
  Admitted.


Lemma term\_erase: forall t, term t -> erase(t) = t.
Proof.
  Admitted.


Lemma body\_erase: forall t, body t -> erase(t) = t.
Proof.
  Admitted.


  esconde código








Theorem strip\_lemma: forall  t t1 t2, t -->B t1 -> t -->>B t2 -> exists t3, t1 -->>B t3 /\symbol{92} t2 -->>B t3.
  Proof.
  intros t t1 t2 H1 H2.
  induction H1.

\begin{itemize}
\item  inversion H; subst.
    apply refltrans\_equiv in H2.
    remember (t1 \^{}\^{} u) as b.
    generalize  dependent b.
    induction H2.
    + exists (t1 \^{}\^{} u); split.
      * apply reflex.
      * apply atleast1.
        apply redex.
        assumption.
    + inversion H; subst.
      rewrite <- H3 in H.
      clear H0 H1 H3.


      inversion H2; subst.
      * admit.
      * admit.
      * admit.
    + 
      apply open\_rec\_inj in H3.
      destruct H3 as \coqdocvar{Heq1} \coqdocvar{Heq2}; subst.
      clear H4 H6.
      induction H2.
      * Admitted.

\end{itemize}






    Lemma erase\_lbeta\_2313: forall t1 t2, t1 -->>B t2 -> forall t1' t2', erase(t1') = t1 /\symbol{92} erase(t2') = t2 -> t1' -->>lB t2'.
    Proof.
      Admitted.


    assert (H': pterm\_app (pterm\_labs t1) u -->>B t2).
    \{
      apply erase\_lbeta\_2313.
    \}
     in H2 with (pterm\_app (pterm\_labs t1) u) \_.




  assert (H2' := H2).
  assert (forall t', erase(t') = t -> forall t2', erase(t2') = t2 -> t' -->>lB t2').
  \{
    apply erase\_lbeta; assumption.   
  \}
  inversion H1; subst.

\begin{itemize}
\item  inversion H0; subst.
    assert (erase (pterm\_app (pterm\_labs t0) u) = pterm\_app (pterm\_abs t0) u -> forall t2', erase t2' = t2 -> pterm\_app (pterm\_labs t0) u -->>lB t2').
    \{
      apply H.
    \}
    clear H.
    assert (erase (pterm\_app (pterm\_labs t0) u) = pterm\_app (pterm\_abs t0) u).
    \{
      simpl.
      rewrite body\_erase.

\begin{itemize}
\item  rewrite term\_erase.
        + reflexivity.
        + assumption.

\item  assumption.

\end{itemize}
    \}
    assert (forall t2' : pterm, erase t2' = t2 -> pterm\_app (pterm\_labs t0) u -->>lB t2').
    \{
      apply H5.
      assumption.
    \}
    clear H H5.
    Admitted.

\end{itemize}




\begin{coqdoccode}
\end{coqdoccode}


% No diretório src:
% coqdoc --latex --body-only confl_lambda.v; echo "$(tail -n +6 confl_lambda.tex)" > confl_lambda.tex


\section{Conclusão}

O presente trabalho desenvolveu parte do formalismo necessário para a construção de uma formalização de uma prova de confluência do cálculo lambda que utiliza duas classes de variáveis: índices de DeBruijn para variáveis ligadas, e nomes para variáveis livres. Esta formalização está sendo construída no assistente de provas Coq, uma ferramenta de código aberto e baseada em uma lógica de ordem superior conhecida como cálculo de construções indutivas.

\bibliographystyle{plain}
\bibliography{references.bib}

\end{document}
